% Options for packages loaded elsewhere
\PassOptionsToPackage{unicode}{hyperref}
\PassOptionsToPackage{hyphens}{url}
%
\documentclass[
  ignorenonframetext,
]{beamer}
\usepackage{pgfpages}
\setbeamertemplate{caption}[numbered]
\setbeamertemplate{caption label separator}{: }
\setbeamercolor{caption name}{fg=normal text.fg}
\beamertemplatenavigationsymbolsempty
% Prevent slide breaks in the middle of a paragraph
\widowpenalties 1 10000
\raggedbottom
\setbeamertemplate{part page}{
  \centering
  \begin{beamercolorbox}[sep=16pt,center]{part title}
    \usebeamerfont{part title}\insertpart\par
  \end{beamercolorbox}
}
\setbeamertemplate{section page}{
  \centering
  \begin{beamercolorbox}[sep=12pt,center]{part title}
    \usebeamerfont{section title}\insertsection\par
  \end{beamercolorbox}
}
\setbeamertemplate{subsection page}{
  \centering
  \begin{beamercolorbox}[sep=8pt,center]{part title}
    \usebeamerfont{subsection title}\insertsubsection\par
  \end{beamercolorbox}
}
\AtBeginPart{
  \frame{\partpage}
}
\AtBeginSection{
  \ifbibliography
  \else
    \frame{\sectionpage}
  \fi
}
\AtBeginSubsection{
  \frame{\subsectionpage}
}
\usepackage{amsmath,amssymb}
\usepackage{lmodern}
\usepackage{iftex}
\ifPDFTeX
  \usepackage[T1]{fontenc}
  \usepackage[utf8]{inputenc}
  \usepackage{textcomp} % provide euro and other symbols
\else % if luatex or xetex
  \usepackage{unicode-math}
  \defaultfontfeatures{Scale=MatchLowercase}
  \defaultfontfeatures[\rmfamily]{Ligatures=TeX,Scale=1}
\fi
% Use upquote if available, for straight quotes in verbatim environments
\IfFileExists{upquote.sty}{\usepackage{upquote}}{}
\IfFileExists{microtype.sty}{% use microtype if available
  \usepackage[]{microtype}
  \UseMicrotypeSet[protrusion]{basicmath} % disable protrusion for tt fonts
}{}
\makeatletter
\@ifundefined{KOMAClassName}{% if non-KOMA class
  \IfFileExists{parskip.sty}{%
    \usepackage{parskip}
  }{% else
    \setlength{\parindent}{0pt}
    \setlength{\parskip}{6pt plus 2pt minus 1pt}}
}{% if KOMA class
  \KOMAoptions{parskip=half}}
\makeatother
\usepackage{xcolor}
\newif\ifbibliography
\usepackage{graphicx}
\makeatletter
\def\maxwidth{\ifdim\Gin@nat@width>\linewidth\linewidth\else\Gin@nat@width\fi}
\def\maxheight{\ifdim\Gin@nat@height>\textheight\textheight\else\Gin@nat@height\fi}
\makeatother
% Scale images if necessary, so that they will not overflow the page
% margins by default, and it is still possible to overwrite the defaults
% using explicit options in \includegraphics[width, height, ...]{}
\setkeys{Gin}{width=\maxwidth,height=\maxheight,keepaspectratio}
% Set default figure placement to htbp
\makeatletter
\def\fps@figure{htbp}
\makeatother
\usepackage[normalem]{ulem}
\setlength{\emergencystretch}{3em} % prevent overfull lines
\providecommand{\tightlist}{%
  \setlength{\itemsep}{0pt}\setlength{\parskip}{0pt}}
\setcounter{secnumdepth}{-\maxdimen} % remove section numbering
\ifLuaTeX
  \usepackage{selnolig}  % disable illegal ligatures
\fi
\IfFileExists{bookmark.sty}{\usepackage{bookmark}}{\usepackage{hyperref}}
\IfFileExists{xurl.sty}{\usepackage{xurl}}{} % add URL line breaks if available
\urlstyle{same} % disable monospaced font for URLs
\hypersetup{
  pdftitle={Présentation de l'outil Podcasts Tracker},
  pdfauthor={Lyna BENYAHIA, Jason MOREL},
  hidelinks,
  pdfcreator={LaTeX via pandoc}}

\title{Présentation de l'outil Podcasts Tracker}
\author{Lyna BENYAHIA, Jason MOREL}
\date{2023-03-07}

\begin{document}
\frame{\titlepage}

\begin{frame}{Introduction}
\protect\hypertarget{introduction}{}
\begin{itemize}
\tightlist
\item
  Constat : il n'existe pas de moyens pour filtrer les résultats par
  durée sur Spotify
\item
  Création d'un outil pour chercher des épisodes/émissions par durée
\item
  Démonstration de l'outil
\end{itemize}
\end{frame}

\begin{frame}{Mise en place du projet}
\protect\hypertarget{mise-en-place-du-projet}{}
\begin{enumerate}
\item
  Utilisation de Spotify pour notre projet à l'aide du package Python
  Spotipy\\
  \uline{Fonction principale que nous avons utilisé} :

  \includegraphics[width=2.17708in,height=\textheight]{images/Capture d’écran 2023-03-07 à 21.37.31.png}

  \uline{NB} : distinction entre émission et épisode, impossible de
  faire des recherches par durée même depuis l'API
\end{enumerate}
\end{frame}

\begin{frame}{Mise en place du projet}
\protect\hypertarget{mise-en-place-du-projet-1}{}
\begin{enumerate}
\setcounter{enumi}{1}
\item
  Création d'un Bot Telegram pour envoyer les liens directement sur le
  téléphone de l'utilisateur

  \includegraphics[width=2.625in,height=\textheight]{images/Capture d’écran 2023-03-07 à 21.48.48.png}

  \begin{itemize}
  \tightlist
  \item
    Récupération du Token afin de connecter le code Python au Bot
    Telegram
  \end{itemize}
\end{enumerate}
\end{frame}

\begin{frame}{Premier problème rencontré}
\protect\hypertarget{premier-probluxe8me-rencontruxe9}{}
\begin{itemize}
\item
  Documentation complexe sur l'API Telegram, manque de temps pour créer
  un bot complètement automatisé\\
  ---\textgreater{} Solution : inputs seront saisis depuis l'ordinateur
  et seuls les résultats seront envoyés sur Telegram

  \includegraphics[width=4.07292in,height=\textheight]{images/Capture d’écran 2023-03-11 à 10.36.02.png}

  \begin{itemize}
  \tightlist
  \item
    Fonction pour envoyer un message sur Telegram (TelegramMessages.py
    sur Github)
  \end{itemize}
\end{itemize}
\end{frame}

\begin{frame}{Interface graphique}
\protect\hypertarget{interface-graphique}{}
\begin{itemize}
\tightlist
\item
  Création d'une interface graphique à l'aide du package Tkinter
  (MessageVersion.py sur Github) :
\end{itemize}

\includegraphics[width=3.70833in,height=\textheight]{images/Capture d’écran 2023-03-11 à 10.10.40.png}
\end{frame}

\begin{frame}{Traitement des émissions}
\protect\hypertarget{traitement-des-uxe9missions}{}
\begin{itemize}
\tightlist
\item
  Show\_treatment sur Github :
\end{itemize}
\end{frame}

\begin{frame}{Traitement des épisodes}
\protect\hypertarget{traitement-des-uxe9pisodes}{}
\begin{itemize}
\tightlist
\item
  Episode\_treatment sur Github :
\end{itemize}

\includegraphics{images/Capture d’écran 2023-03-11 à 10.29.03.png}
\end{frame}

\begin{frame}{Conclusion}
\protect\hypertarget{conclusion}{}
\begin{itemize}
\tightlist
\item
  Pratique que chacun travaille sur une partie différente du code
\item
  MAIS, cela complique l'assemblage des différentes parties
\item
  Extension possible :

  \begin{itemize}
  \tightlist
  \item
    Bot Telegram complètement autonome
  \item
    Donner plus de résultats sur un même sujet
  \item
    Garder en mémoire chat\_id
  \item
    Créer .log des recherches et résultats de l'utilisateur
  \end{itemize}
\end{itemize}
\end{frame}

\end{document}
